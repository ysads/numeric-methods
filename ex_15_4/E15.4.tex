\documentclass[11pt]{amsart}

%%%% Pacotes adicionais para enumeração de items e alinhamento de equações
%%%%
\usepackage{enumitem}
\usepackage{amsmath}
\usepackage{mathtools}
\usepackage{systeme}

%%%% Escrevendo em português
%%%% 
\usepackage[utf8]{inputenc}

%%%% Configurações para output correto de pdf
%%%%

\usepackage[T1]{fontenc}
\usepackage{ae}
\usepackage{aecompl}
\usepackage{amssymb}

%%%% Layout de página
%%%%
\usepackage{fullpage}
\usepackage{setspace}
\usepackage{tabu}

%%%% Comandos personalizados
%%%%
\newcommand\numberthis{\addtocounter{equation}{1}\tag{\theequation}}
\newcommand\blockskip{\bigskip\medskip}

%%%% Define a altura das linhas
\linespread{1.15}

\begin{document}
\parindent=0pt

\title[MAC0210]
{MAC0210 Laboratório de Métodos Numéricos}\vspace{3\jot}%\\\vspace{1\jot}

\footskip=28pt

\maketitle
\pagestyle{plain}

\textbf{Nome: Ygor Sad Machado}\hfill
\textbf{Número USP: 8910368}\null

\medskip
\textbf{Exercício 15.4}\hfill
\textbf{Data: 30/06/2020}\null

\rule{\textwidth}{0.4pt}

\textbf{a.}
Queremos estimar o erro na regra trapezoidal corrigida e mostrar que ele pode ser estimado por

\begin{equation}
    E(f) = \frac{f^{(4)}(\xi)}{720} (b-a)^5
\end{equation}

\bigskip
Seja $p_k(x) = \sum_{j=0}^k a_j f(x_j)$ uma aproximação polinomial para $f$. Pela definição, podemos escrever tal erro como:

\begin{align*}
    E(f) &= \int_a^b f(x) dx - \int_a^b p_k(x) dx\\
         &= \int_a^b f(x) - p_k(x) dx
\end{align*}

\bigskip
No entanto, sabemos que o erro na interpolação polinomial é dado por:

\begin{equation}
    f(x) - p_k(x) = f[x_0, \dots, f_k, x] \prod_{i=0}^k (x - x_i)
\end{equation}

\bigskip
Logo, podemos escrever:

\begin{align}
    E(f) &= \int_a^b f[x_0, \dots, f_k, x] \prod_{i=0}^k (x - x_i) dx \label{expanded_error}
\end{align}

\bigskip
Usando o teorema do valor médio para diferenças divididas, descobrimos que:

\begin{equation}
    f[x_0, \dots, f_k, x] = \frac{f^{(k+1)}(\xi)}{(k+1)!}, \quad \text{ para } \xi \in [a, b] \label{mean_diff}
\end{equation}

\bigskip
Utilizando \eqref{mean_diff} em \eqref{expanded_error} teremos, em particular, para o nosso contexto:

\begin{align*}
    E(f) &= \int_a^b f[a, a, b, b, x] (x-a)^2(x-b)^2 dx\\
         &= f[a, a, b, b, \xi] \int_a^b (x-a)^2(x-b)^2 dx\\
         &= \frac{f^{(4)}(\xi)}{4!} \int_a^b (x-a)^2(x-b)^2 dx\\
\end{align*}

\bigskip
Finalmente, expandindo a integral e simplificando o resultado chegamos à forma final de $E(f)$:

\begin{align*}
    E(f) &= \frac{f^{(4)}(\xi)}{24} \Big(-\frac{1}{30}(a-b)^5\Big)\\
         &= \frac{f^{(4)}(\xi)}{720} (b-a)^5
\end{align*}
\qed\null

\bigskip
\bigskip
\rule{\textwidth}{0.4pt}
\textbf{b.}
Foi incluído, junto a este PDF, o script octave \verb|E15.4m|, que contém a simulação geradora dos dados abaixo mencionados.

\bigskip
Em $\int_0^1 e^x dx$, percebe-se que o método trapezoidal corrigido se aproxima mais do que o \textit{mid-rule} e o trapezoidal básico, mas ainda assim é menos preciso que o método de Simpson.

\bigskip
\begin{center}
    \begin{tabular}{c|c}
        \hline
        Método & Estimativa de $I(f)$ \\
        \hline
        \textbf{Valor exato} & 1.7183...\\
        Trapezoidal corrigido & 1.7159...\\
        Trapezoidal básico& 1.8591... \\
        Simpson & 1.7189... \\
        Mid-rule & 1.6487... \\
        \hline
    \end{tabular}
\end{center}

\bigskip
\bigskip
Novamente, em $\int_0.9^1 e^x dx$, percebe-se que o método trapezoidal corrigido se aproxima do valor real mais do que o \textit{mid-rule} e o trapezoidal básico, ficando, no entanto, aquém do método de Simpson. É importante mencionar que, conforme esperado, o intervalo mais curto tem como consequência precisão mais alta na estimativa.

\bigskip
\begin{center}
    \begin{tabular}{c|c}
        \hline
        Método & $|E(f)|$ \\
        \hline
        Trapezoidal corrigido & $2.2 \times 10^{-4}$\\
        Trapezoidal básico & $3.59 \times 10^{-8}$ \\
        Simpson & $9.0 \times 10^{-9}$ \\
        Mid-rule & $1.1 \times 10^{-4}$ \\
        \hline
    \end{tabular}
\end{center}
\qed\null

\end{document}