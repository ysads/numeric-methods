\documentclass[11pt]{amsart}

%%%% Pacotes adicionais para enumeração de items e alinhamento de equações
%%%%
\usepackage{enumitem}
\usepackage{amsmath}
\usepackage{mathtools}
\usepackage{systeme}

%%%% Escrevendo em português
%%%% 
\usepackage[utf8]{inputenc}

%%%% Configurações para output correto de pdf
%%%%

\usepackage[T1]{fontenc}
\usepackage{ae}
\usepackage{aecompl}
\usepackage{amssymb}

%%%% Layout de página
%%%%
\usepackage{fullpage}
\usepackage{setspace}
\usepackage{tabu}

%%%% Comandos personalizados
%%%%
\newcommand\numberthis{\addtocounter{equation}{1}\tag{\theequation}}
\newcommand\blockskip{\bigskip\medskip}

%%%% Define a altura das linhas
\linespread{1.15}

\begin{document}
\parindent=0pt

\title[MAC0210]
{MAC0210 Laboratório de Métodos Numéricos}\vspace{3\jot}%\\\vspace{1\jot}

\footskip=28pt

\maketitle
\thispagestyle{empty} 
\pagestyle{plain}

\textbf{Nome: Ygor Sad Machado}\hfill
\textbf{Número USP: 8910368}\null

\medskip
\textbf{Exercício 14.3}\hfill
\textbf{Data: 20/06/2020}\null

\rule{\textwidth}{0.4pt}
\medskip

Foi incluído, junto a este PDF, o script octave \verb|E14.3.m| que contém uma simulação do método descrito abaixo para $x_0 = 0.4$ e $h =$ 0.1, 0.5 e 0.25. O objetivo é verificar que o método fornece um resultado com precisão suficiente.

\bigskip
\rule{\textwidth}{0.4pt}
\medskip

Queremos derivar uma fórmula para aproximar a segunda derivada de uma função $f(x)$ suficientemente suave, usando expansões de Taylor de sexta ordem.

\medskip
Para isso, assuma que possuímos os sete pontos $x_0 - 3h, x_0 - 2h, x_0 - h, x_0, x_0 + h, x_0 + 2h, x_0 + 3h$. Vamos escrever $f''(x_0)$ em forma da combinação linear das expansões de Taylor ao redor destes sete pontos. Isto é:

\begin{align*}
    f''(x_0) &= Af(x_0 - 3h) + Bf(x_0 - 2h) + Cf(x_0 - 1h)\\ 
             &+ Df(x_0) + Ef(x_0 + h) + Ff(x_0 + 2h) + Gf(x_0 + 3h) \numberthis\label{f_2_linear_comb}\\
\end{align*}

\blockskip
Expandindo a equação acima usando as expansões de Taylor obtemos:

{\scriptsize
\begin{align*}
    f''(x_0) &= A\Big[ f(x_0)
                       - 3hf'(x_0)
                       + \frac{(3h)^2}{2}f''(x_0)
                       - \frac{(3h)^3}{6}f'''(x_0)
                       + \frac{(3h)^4}{24}f^{(4)}(x_0)
                       - \frac{(3h)^5}{120}f^{(5)}(x_0)
                       + \frac{(3h)^6}{720}f^{(6)}(x_0)
                       + \mathcal{O}(h^6) \Big]\\
             &+ B\Big[ f(x_0)
                       - 2hf'(x_0)
                       + \frac{(2h)^2}{2}f''(x_0)
                       - \frac{(2h)^3}{6}f'''(x_0)
                       + \frac{(2h)^4}{24}f^{(4)}(x_0)
                       - \frac{(2h)^5}{120}f^{(5)}(x_0)
                       + \frac{(2h)^6}{720}f^{(6)}(x_0)
                       + \mathcal{O}(h^6) \Big]\\
             &+ C\Big[ f(x_0)
                       - hf'(x_0)
                       + \frac{h^2}{2}f''(x_0)
                       - \frac{h^3}{6}f'''(x_0)
                       + \frac{h^4}{24}f^{(4)}(x_0)
                       - \frac{h^5}{120}f^{(5)}(x_0)
                       + \frac{h^6}{720}f^{(6)}(x_0)
                       + \mathcal{O}(h^6) \Big]\\
             &+ D\Big[ f(x_0) \Big]\\
             &+ E\Big[ f(x_0)
                       + hf'(x_0)
                       + \frac{h^2}{2}f''(x_0)
                       + \frac{h^3}{6}f'''(x_0)
                       + \frac{h^4}{24}f^{(4)}(x_0)
                       + \frac{h^5}{120}f^{(5)}(x_0)
                       + \frac{h^6}{720}f^{(6)}(x_0)
                       + \mathcal{O}(h^6) \Big]\\
             &+ F\Big[ f(x_0)
                       + 2hf'(x_0)
                       + \frac{(2h)^2}{2}f''(x_0)
                       + \frac{(2h)^3}{6}f'''(x_0)
                       + \frac{(2h)^4}{24}f^{(4)}(x_0)
                       + \frac{(2h)^5}{120}f^{(5)}(x_0)
                       + \frac{(2h)^6}{720}f^{(6)}(x_0)
                       + \mathcal{O}(h^6) \Big]\\
             &+ G\Big[ f(x_0)
                       + 3hf'(x_0)
                       + \frac{(3h)^2}{2}f''(x_0)
                       + \frac{(3h)^3}{6}f'''(x_0)
                       + \frac{(3h)^4}{24}f^{(4)}(x_0)
                       + \frac{(3h)^5}{120}f^{(5)}(x_0)
                       + \frac{(3h)^6}{720}f^{(6)}(x_0)
                       + \mathcal{O}(h^6) \Big]\\
\end{align*}
}%

\pagebreak
Rearranjando os termos da equação acima de maneira mais conveniente, obtemos:

{\small
\begin{align*}
f''(x_0) &=\ (A + B + C + D + E + F + G)\ f(x_0)\\
         &+\ (-3A - 2B - C + E + 2F + 3G)\ hf'(x_0)\\
         &+\ (9A + 4B + C + E + 4F + 9G)\ \frac{h^2}{2}f''(x_0)\\
         &+\ (-27A - 8B - C + E + 8F + 27G)\ \frac{h^3}{6}f'''(x_0)\\
         &+\ (81A + 16B + C + E + 16F + 81G)\ \frac{h^4}{24}f^{(4)}(x_0)\\
         &+\ (-243A - 32B - C + E + 32F + 243G)\ \frac{h^5}{120}f^{(5)}(x_0)\\
         &+\ (729A + 64B + C + E + 64F + 729G)\ \frac{h^6}{720}f^{(6)}(x_0)\\
\end{align*}
}%

\blockskip
Igualando os coeficientes dos dois lados da equação, temos um sistema de 7 equação e 7 incógnitas:

{\small
\begin{align*}
    0 &= A + B + C + D + E + F + G\\
    0 &= (-3A - 2B - C + E + 2F + 3G)h\\
    1 &= (9A + 4B + C + E + 4F + 9G)\frac{h^2}{2}\\
    0 &= (-27A - 8B - C + E + 8F + 27G)\frac{h^3}{6}\\
    0 &= (81A + 16B + C + E + 16F + 81G)\frac{h^4}{24}\\
    0 &= (-243A - 32B - C + E + 32F + 243G)\frac{h^5}{120}\\
    0 &= (729A + 64B + C + E + 64F + 729G)\frac{h^6}{720}\\
\end{align*}
}%

\blockskip
Finalmente, resolvendo o sistema acima, chegamos a:

{\small
\begin{equation*}
    A = \frac{1}{90h^2} \qquad
    B = -\frac{3}{20h^2} \qquad
    C = \frac{3}{2h^2} \qquad
    D = -\frac{49}{18h^2} \qquad
    E = \frac{3}{2h^2} \qquad
    F = -\frac{3}{20h^2} \qquad
    G = \frac{1}{90h^2} \qquad
\end{equation*}
}%

\blockskip
\medskip
\medskip
Finalmente, aplicando as incógnitas acima em \eqref{f_2_linear_comb}, obtemos uma expressão para $f''(x)$:

{\scriptsize
\begin{equation*}
    f''(x_0) = \frac{1}{90h^2}f(x_0 - 3h)
             - \frac{3}{20h^2}f(x_0 - 2h)
             + \frac{3}{2h^2}f(x_0 - h)
             - \frac{49}{18h^2}f(x_0)
             + \frac{3}{h^2}f(x_0 + h)
             - \frac{3}{20h^2}f(x_0 + 2h)
             + \frac{1}{90h^2}f(x_0 + 3h)
\end{equation*}
}%
\qed\null

\end{document}